\documentclass[12pt]{article}
\usepackage{amsmath, amssymb, amsthm, graphicx}

\title{APPM 4600 \\ Homework 2}
\author{Dani Lisle | STID:109 97 2839}
\date{\today}

\begin{document}
\maketitle

\section*{Problem 1}
\subsection*{Part a}

Given the binomial expansion,
\begin{equation*}
(1 + x)^n = 1 + n x + \frac{n(n-1)}{2!}x^2 + \ldots
\end{equation*}
As \(x \rightarrow 0\), the higher-order terms grow in total at a significantly lower rate then $x$, so we have
\begin{equation*}
(1 + x)^n = 1 + n x + o(x).
\end{equation*}

\newpage

\subsection*{Part b}

To show that $x \sin(\sqrt{x}) = O(x^{3/2})$ as $x \rightarrow 0$, we consider the function $f(x) = x \sin(\sqrt{x})$ and compare its growth rate to that of $g(x) = x^{3/2}$. We evaluate the derivatives of both functions:

\[
f'(x) = \frac{d}{dx}\left( x \sin(\sqrt{x}) \right) = \frac{\sqrt{x} \cos(\sqrt{x})}{2} + \sin(\sqrt{x})
\]

\[
g'(x) = \frac{d}{dx}\left( x^{3/2} \right) = 1.5x^{0.5}
\]

We then examine the limit of the ratio of these derivatives as $x$ approaches 0:

\[
\lim_{x \to 0} \frac{f'(x)}{g'(x)} = \lim_{x \to 0} \frac{\frac{\sqrt{x} \cos(\sqrt{x})}{2} + \sin(\sqrt{x})}{1.5x^{0.5}}
\]

Evaluating this limit yields:

\[
\lim_{x \to 0} \frac{f'(x)}{g'(x)} = 1
\]

This result indicates that the growth rate of $f(x)$ is comparable to that of $g(x)$ as $x$ approaches 0. Additionally, $f$'s growth slows as $x$ increases close to $0$, so $x \sin(\sqrt{x}) = O(x^{3/2})$ as $x \rightarrow 0$.

\subsection*{Part c}

To show that \(e^{-t} = o\left(\frac{1}{t^2}\right)\) as \(t \rightarrow \infty\), we evaluate the limit of the ratio of \(e^{-t}\) to \(\frac{1}{t^2}\) as \(t\) approaches infinity:

\[
\lim_{t \to \infty} \frac{e^{-t}}{\frac{1}{t^2}} =  \lim_{t \to \infty} \frac{t^2}{e^t} = \lim_{t \to \infty} \frac{t^2}{e^t} = \lim_{t \to \infty} \frac{2}{e^t} = 0
\]

After repeated applications of L'Hopital's Rule, the exponential term \(e^{-t}\) in the numerator ensures that the limit approaches 0, as the exponential decay outpaces any polynomial decay, satisfying the condition for \(e^{-t} = o\left(\frac{1}{t^2}\right)\).


\subsection*{Part d}



Bounded by a constant multiple of $\epsilon$.

\newpage

\section*{Problem 2}

\subsection*{Part a}

The change in the solution $\Delta x$ due to the perturbation in $b$ can be calculated using the formula $\Delta x = A^{-1} \Delta b$:

\[
\Delta x = 
\begin{bmatrix}
1 - 10^{10} & 10^{10} \\
-10^{10} & 1 + 10^{10}
\end{bmatrix}
\begin{bmatrix}
\Delta b_1 \\
\Delta b_2
\end{bmatrix}
=
\begin{bmatrix}
-(10^{10} - 1)\Delta b_1 + 10^{10}\Delta b_2 \\
-10^{10}\Delta b_1 + (10^{10} + 1)\Delta b_2
\end{bmatrix}
\]






\end{document}